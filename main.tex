\documentclass[conference]{IEEEtran}
\IEEEoverridecommandlockouts

\usepackage{algorithmic}
\usepackage{amsmath, amssymb, amsfonts}
\usepackage{cite}
\usepackage{float}
\usepackage{graphicx}
\usepackage{hyperref}
\usepackage{textcomp}
\usepackage{xcolor}

\def\BibTeX{{\rm B\kern-.05em{\sc i\kern-.025em b}\kern-.08em
    T\kern-.1667em\lower.7ex\hbox{E}\kern-.125emX}}

\begin{document}

\title{Comparative Study of Medical Information Retrieval Approaches}

\graphicspath{ {./images/} }

\makeatletter
\newcommand{\linebreakand}{%
  \end{@IEEEauthorhalign}
  \hfill\mbox{}\par
  \mbox{}\hfill\begin{@IEEEauthorhalign}
}
\makeatother

\author{
    \IEEEauthorblockN{Davin Jason Evan Raharjo}
    \IEEEauthorblockA{\textit{Department of Computer Science} \\
    \textit{and Electronics} \\
    \textit{Universitas Gadjah Mada} \\
    Yogyakarta, Indonesia \\
    davinjasonevanraharjo@mail.ugm.ac.id}
    \and
    \IEEEauthorblockN{Asyraf Nur Ardliansyah}
    \IEEEauthorblockA{\textit{Department of Computer Science} \\
    \textit{and Electronics} \\
    \textit{Universitas Gadjah Mada} \\
    Yogyakarta, Indonesia \\
    asyrafnurardliansyah@mail.ugm.ac.id}
    \and
    \IEEEauthorblockN{Isneyri Arsyadani}
    \IEEEauthorblockA{\textit{Department of Computer Science} \\
    \textit{and Electronics} \\
    \textit{Universitas Gadjah Mada} \\
    Yogyakarta, Indonesia \\
    isneyriarsyadani@mail.ugm.ac.id}
}

\maketitle

\begin{abstract}
    soon
\end{abstract}

\begin{IEEEkeywords}
    soon
\end{IEEEkeywords}

\section{Introduction}

The increasing demand for accessible health information has made Medical Information Retrieval (IR) a crucial field in contemporary healthcare. As healthcare decisions increasingly rely on information availability, both healthcare professionals and the general public seek efficient systems to facilitate this access. Medical IR plays a pivotal role in supporting these needs by providing tailored information that can guide decision-making and health education. The evolution of medical IR has seen a shift towards user-centered approaches, emphasizing the importance of addressing the diverse requirements of different user groups, ranging from patients to specialized clinicians.

Recent advancements in medical IR research have focused on enhancing the relevance and accuracy of retrieved information through innovative methodologies. Among these, graph-based models have emerged as a powerful tool for representing complex medical knowledge, enabling systems to understand relationships between various medical concepts. Furthermore, techniques in terminology extraction are being utilized to improve the retrieval of domain-specific information. Clinical decision support systems, which integrate data and algorithms to assist healthcare professionals, also represent a significant stride towards optimizing medical IR. These developments underline the importance of creating personalized IR systems that take into account the unique contexts and needs of users.

Despite the promising advancements in medical IR, several challenges remain that must be addressed to ensure effective implementation. Diverse user needs pose a significant hurdle, as different users require varying levels of detail and types of information. Additionally, the varying medical knowledge of users necessitates that information is presented in an accessible manner, catering to both experts and laypeople. Language barriers further complicate this landscape, as much of the medical content is predominantly in English, limiting access for non-English speakers. Moreover, the complexity of health records, which often consist of varied formats such as clinical notes and pathology data, requires sophisticated retrieval methods. Finally, ensuring the reliability and accuracy of retrieved medical information is critical, as it directly impacts health decisions. Addressing these challenges is essential for the advancement of Medical IR and its ability to improve healthcare outcomes.

\bibliographystyle{ieeetr}
\bibliography{references}

\end{document}
